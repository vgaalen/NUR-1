\newpage
\section{Vandermonde Matrix}
For this exercise, I implemented a function to solve a system of linear equations
using LU decomposition. This was then applied to the Vandermonde matrix to
find a 19th degree Lagrange polynomial for the provided dataset.

\subsection*{2a}
Figure \ref{fig:vandermonde} shows the result of the LU decomposition, implemented using Crout's algorithm.
\begin{figure}
    \centering
    \includegraphics[width=0.8\textwidth]{plots/vandermonde.png}
    \caption{The 19th degree Lagrange polynomial (evaluated for 1000 points 
    between highest- and lowest x-values in the data) and the datapoints it 
    was fitted to. The lower plots indicate the absolute value of the 
    deviation of the polynomial to the datapoints and the right two plots are zoom-ins of the left two.}
    \label{fig:vandermonde}
\end{figure}

\noindent
This is the code used for this exercise:
\lstinputlisting[language=python,firstline=1, lastline=103]{vandermonde.py}

\subsection*{2b}
Here we compare the polynomial found using LU decomposition to one found using Neville's algorithm.
The polynomial found using Neville's algorithm deviates significantly
from the one found in (2a) in a number of places (see Figure \ref{fig:neville}). 
On the intervals where the LU decomposition polynomial is well behaved, the 
Neville polynomial is well behaved as well. In places where the LU 
decomposition polynomial finds values far higher or lower than the 
datapoints around the interval, the Neville polynomial does worse in two 
cases and better in two others. This difference is due to the way these 
algorithms handle the edge cases, where there are not enough datapoints to 
do find a genuine 19th degree polynomial.
At the x-values of the actual datapoints, Nevilles algorithm is able to find y-values much closer to the actual data 
than the LU decomposition (see the bottom half of \ref{fig:neville}).

\begin{figure}
    \centering
    \includegraphics[width=0.8\textwidth]{plots/neville_compare.png}
    \caption{Comparison between the 19th degree Lagrange polynomial found
    using Neville's algorithm and the one found using LU decomposition. The 
    19th degree Lagrange polynomial was evaluated for 1000 points between 
    highest- and lowest x-values in the data and plotted with the datapoints 
    it was fitted to. The black text indicates the absolute value of the 
    deviation of the Neville polynomial to the datapoints.}
    \label{fig:neville}
\end{figure}

\noindent
This is the code used for this exercise:
\lstinputlisting[language=python, firstline=106, lastline=212]{vandermonde.py}

\subsection*{2c}
In this exercise, we itterate on the LU decomposition done in (2a) to improve
the solution. This is done by solving the system for the error found in the
previous itteration. This should allow us to improve the solution as, if $c=c\mathaccent+\delta c$,
then $(A c)=(A c')+(A \delta c)$.
Figure \ref{fig:lu_itt} shows the result of this itterative improvement, with
snapshots at 1 and 10 itterations.
It turns out that these itteration do not reduce the error. This makes sense for the low x-values, where the error is on the
order of roundoff error for float64, but it is surprising that there is no improvement compared to datapoints at higher x-values,
as the error in the first itteration is relatively high here.

\begin{figure}
    \centering
    \includegraphics[width=0.8\textwidth]{plots/vandermonde_itt.png}
    \caption{The 19th degree Lagrange polynomial derived using LU 
    decomposition (left) and itterative improvents thereon, 1 and 10 
    itterations in the center and on the right resp. (evaluated for 1000 
    points between highest- and lowest x-values in the data). This was 
    plotted with the datapoints it was fitted to. The black text indicates 
    the absolute value of the deviation of the polynomial to the datapoints.}
    \label{fig:lu_itt}
\end{figure}

\subsection*{2d}
For this question we compare the time it takes to generate the polynomials
discussed in (2a)-(2c). When averaging over 10 runs we get the following:

\lstinputlisting{vandermonde_timing.txt}

\noindent
This shows that this implementation of Crout's algorithm and solving using
LU decomposition is faster than this implementation of Neville's algorithm.
Furthermore, it shows how much faster the itterative improvement is 
to the first solution. This is because the LU decomposition only has to be
performed once, suggesting that this operation is the bottleneck of the 
algorithm.

\noindent
This is the code used in this exercise:
\lstinputlisting[language=python]{vandermonde.py}