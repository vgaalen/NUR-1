\section{Poisson Distribution}

The main challenge of this exercise is to find a method to calculate the 
poisson distribution for sufficiently high values of \lambda and k, without 
exceeding the range of a 32-bit floating point number.
I found two methods to do so. The first revolves around itteratively applying
the $\lambda^k$ multiplication and $k!$ division. The second operates in
log-space, and therefore performs operations on far smaller numbers.
It turned out that the second approach is a factor 2 faster over the set of
values for \lambda and k provided.

\lstinputlisting[caption=Time time it takes to perform exercise 1a using the two approaches, averaged over 1000 runs]{poisson_timing.txt}

This yielded the following values:\\
\lstinputlisting{poisson.txt}

Code used for this question:
\lstinputlisting{poisson.py}


